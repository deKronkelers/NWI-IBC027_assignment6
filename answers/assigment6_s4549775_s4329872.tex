\documentclass[12pt]{article}

\usepackage{amssymb}
\usepackage{multicol}
\usepackage{enumerate}
\usepackage[top=5em, bottom=5em, left=5em, right=5em]{geometry}
\usepackage{listings}
\usepackage{tikz}
\usetikzlibrary{positioning}

\setlength\parindent{0em}
\setlength\parskip{1em}

\title {Assignment 6}

\author {Hendrik Werner s4549775}

\begin{document}
\maketitle

This was done in collaboration with Constantin Blach (s4329872).

\section{} %1
\section{} %2
We came up with the following algorithm:

\begin{lstlisting}
G' := G

for vertex in G'.V
	if capacity(vertex.incoming) > l(vertex)
		inc := vertex.incoming
		out := vertex.outgoing
		G'.V = G'.V - vertex
		vertex_in := new Vertex(incoming=inc)
		vertex_out := new Vertex(outgoing=out)
		G'.V := G'.V + {vertex_in, vertex_out}
		edge := new Edge(
			from=vertex_in
			,to=vertex_out
			,capacity=l(vertex)
		)
		G'.E := G'.E + edge
\end{lstlisting}

Where $l(v)$ returns the maximum flow through vertex $v$, $v.incoming$ are the incoming edges to vertex $v$, $v.outgoing$ are the outgoing edges from vertex $v$, and $G.V$ are the vertices of graph $G$.

The algorithm does not change flow through the graph because flows that were limited by $l(v)$ are now limited by the edge between $v\_in$ and $v\_out$.

We iterate over each vertex once ($O(|V|)$) and each edge twice (because each edge has one source and one destination and we touch it once for both: $O(2|E|) = O(|E|)$) and do $O(1)$ work. The total time complexity is $O(|V| + |E|)$.

\section{} %3
We propose this algorithm:

\begin{enumerate}
	\item generate a residual graph $R$ from input graph $G$
	\item find all non cyclic paths $P$ from $t$ to $s$ in $R$ \label{paths}
	\item initialize an empty set of bottleneck edges $B$
	\item for each path $p$ in $P$
		\begin{enumerate}
			\item initialize $candidates := \{\}$
			\item for each edge $e = (from, to)$ in $p$
			\begin{enumerate}
				\item if $flow(e) == G.capacity((to, from))$
				\begin{enumerate}
					\item $candidates := candidates + e$
				\end{enumerate}
			\end{enumerate}
			\item if $candidates$ contains exactly one edge $e$
			\begin{enumerate}
				\item $B := B + e$
			\end{enumerate}
		\end{enumerate}

	(This could be done with BFS.)
\end{enumerate}

$B$ contains all bottleneck edges.

Where $G.capacity(e)$ returns the capacity of edge $e$ in graph $G$.

Correctness: A bottleneck edge is defined as an edge whose capacity limits the maximum flow of the graph. For any path $p$ in $P$ (after step \ref{paths}) there is either one or more edges operating at full capacity. If there is only one edge operating at full capacity then increasing its capacity will result in higher maximum flow. If there is more than one edge operating at full capacity then increasing either of them will not improve the flow as it will still be limited by another edge on the path.

\section{} %4
\section{} %5

\end{document}
